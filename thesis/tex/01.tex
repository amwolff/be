\chapter{Wprowadzenie do fingerprintingu}

\section{Podstawowe pojęcia}

\subsection{Nomenklatura używana w tej pracy}
Pisząc o odcisku palca użyto (także w tytule pracy) ogólnie przyjętego skrótu
myślowego, oznaczającego odbitkę linii papilarnych, czyli formę językową
uznawaną za poprawną przez specjalistów od daktyloskopii.

Użycie formy językowej ,,odcisk palca'' w terminie ,,cyfrowy odcisk palca'' ma
wiele sensu. Jeszcze bez zdefiniowania tego specjalistycznego terminu, możemy
domyślić się, co oznacza. Oczywiście wynika to z faktu, że cyfrowy odcisk palca
i analogowy odcisk palca są ze sobą w pewien sposób powiązane (koncepcja
cyfrowego odcisku palca czerpie z wartości wynikających ze stosowania odbitek
ludzkich linii papilarnych w dziedzinie kryminalistyki).

Angielskie słowo ,,fingerprint'' tłumaczy się jako odcisk palca, jednakże w
zagranicznych publikacjach dotyczących cyfrowego odcisku palca rzadko występuje
termin ,,digital fingerprint''. Jak piszą Flood i Karlsson
\cite[s. 4]{flood2012browser}, kontekst użycia jest na tyle wyraźny, że użycie
samego ,,fingerprint'' jest wyczerpujące.

Zachodnie nazewnictwo ma tę przewagę, że jest zdecydowanie bardziej kompaktowe.
Także w przypadku słowotwórczego zabiegu \emph{fingerprinting}, oznaczającego
czynność; szukając polskiego odpowiednika musielibyśmy sięgnąć po ,,cyfrowe
znakowanie''. Z uwagi na tę kompaktowość i łatwość użycia, w pracy preferowane
będzie użycie oryginalnej nomenklatury.

\subsection{Definicje}
W kolejnych punktach zawarto najważniejsze definicje i powiązane pojęcia,
analogiczne do tych obecnych w literaturze
\cite{eckersley2010unique,flood2012browser}, które będą używane w przeciągu
całej pracy.

\subsubsection{Fingerprint}
Wektor cech pozwalający zidentyfikować dowolny zbiór danych.

Aby \emph{fingerprint} pełnił praktyczną funkcję identyfikacyjną, tak jak
odbitka ludzkich linii papilarnych pełni praktyczną funkcję identyfikacyjną,
często stosuje się algorytm, który kojarzy wektor cech z określonej długości
(zwykle krótkim) ciągiem bajtów (identyfikatorem; można go także rozumieć jako
etykieta). Takim algorytmem może być na przykład wysokiej wydajności funkcja
skrótu (niekoniecznie zdatna do zastosowań kryptograficznych---na przykład
MurmurHash). W niektórych źródłach można także spotkać się z taką definicją, że
\emph{fingerprint} to już sam wynik wyżej wspomnianego algorytmu. Taka definicja
nie zmienia istoty \emph{fingerprintu}, ale jest zdecydowanie mniej przydatna w
kontekście \emph{fingerprintingu} urządzeń podłączonych do Internetu i
przeglądarek internetowych, czego dotyczy niniejsza praca.

\subsubsection{Fingerprint urządzenia podłączonego do Internetu}
Wektor cech pozwalający zidentyfikować urządzenie podłączone do Internetu.

\subsubsection{Instalacja przeglądarki internetowej}
Instalacja na konkretnym urządzeniu. W przypadku zmiany ustawień, konfiguracji i
liczby pluginów / rozszerzeń oraz aktualizacji przeglądarki, instalacja
przeglądarki pozostaje ciągle tą samą instalacją.

\subsubsection{Fingerprint przeglądarki internetowej}
Wektor cech pozwalający zidentyfikować instalację przeglądarki internetowej.

\subsection{Właściwości fingerprintu}
Ludzkie linie papilarne są na ogół niepowtarzalne, niezmienne i nieusuwalne. Z
wartości wynikających ze stosowania ich odbitek w swojej dziedzinie badawczej
czerpie (także etymologicznie) koncepcja \emph{fingerprintu} i dlatego też
\emph{fingerprint} z dobrze dobranymi cechami będzie odzwierciedlać podobne
właściwości.

W przypadku \emph{fingerprintu} urządzeń podłączonych do Internetu i
przeglądarek internetowych, najważniejszymi jego właściwościami są
unikalność/różnorodność (niepowtarzalność) oraz stabilność (niezmienność), przy
czym zwiększenie unikalności lub stabilności ma najczęściej negatywny wpływ na
drugi parametr \cite[s. 11]{eckersley2010unique}.

Jedną ze stosowanych \cite[s. 6]{eckersley2010unique} metod pomiaru unikalności
\emph{fingerprintu} urządzeń i przeglądarek jest entropia Shannona.

\subsubsection{Entropia Shannona}
Wartość entropii można rozumieć jako liczbę pytań binarnych potrzebnych do
sklasyfikowania losowo wybranego elementu z danego zbioru. Zatem entropia
Shannona zbioru \(D\) z etykietami \(\{l_{0}, l_{1}, \dots, l_{n - 1}\}\) wyraża
się wzorem \[H(D) = -{\sum_{i = 0}^{n - 1}{p(l_{i})\log_{2}{p(l_{i})}}}\] gdzie
\(p(l_{i})\) to wyrażona ułamkiem częstość \(x \in D\) mającego etykietę
\(l_{i}\). W przypadku w którym każda etykieta występuje tak samo często
entropia ma wartość maksymalną równą \(\log_{2}{n}\).

\paragraph{Przykład}
Jeśli zbiór \emph{fingerprintów} przeglądarek internetowych ma \(32\) bity
entropii, to w przypadku losowego wyboru jednego z nich oczekujemy, że w
najlepszym przypadku tylko \(1\) na \(4294967295\) przeglądarek będzie miała
taki sam \emph{fingerprint}.

\subsubsection{Stabilność}
W przypadku dodania kolejnej cechy do wektora cech, identyfikującego urządzenie
lub przeglądarkę, zwykle zwiększy to entropię, ale także zmniejszy stabilność
\emph{fingerprintu}. Dzieje się tak, ponieważ jest to kolejna rzecz, która może
zmienić się w czasie. Jeśli jedną z cech wejściowych jest wersja oprogramowania
urządzenia lub wersja przeglądarki (która zwykle zmienia się parę razy w ciągu
roku), to kolejne \emph{fingerprinty} mogą odbiegać od siebie i naiwny
klasyfikator, korzystający z algorytmu reagującego na najmniejsze zmiany (na
przykład funkcja skrótu), mógłby nadać takiemu urządzeniu/przeglądarce kolejną
etykietę, zamiast potraktowania jej jako poprzednio widzianą instalację.

\section{Fingerprinting a Internet} % https://sjp.pwn.pl/poradnia/haslo/;228
Aby lepiej zrozumieć istotę \emph{fingerprintu} i motywację stojącą za
stosowaniem \emph{fingerprintingu} w kontekście urządzeń podłączonych do
Internetu oraz przeglądarek internetowych, wspominając o różnych innych
obszarach przetwarzania komputerowego w których wykorzystywany jest
\emph{fingerprinting} w stosownych mu celach, kolejne punkty posłużą jako
referencja (także historyczna).

\subsection{Początki Internetu}
Początek Internetu jaki znamy obecnie, to początek stworzonej w 1969 roku na
potrzeby amerykańskiego wojska sieci ARPAnet. ARPAnet była implementacją
niezależnych prac Paula Barana, Donalda Daviesa i Leonarda Kleinrocka z lat 60.
XX wieku. Na samym początku swojego istnienia, Internet wykorzystywany był do
tego aby rozpraszać obliczenia pomiędzy wiele komputerów---w tym wypadku
chodziło o superkomputery znajdujące się w innych ośrodkach badawczych (ARPAnet
powstało na Uniwersytecie Kalifornijskim w Los Angeles) \cite{press2015very}. W
tym samym okresie powstawały inne globalne sieci komputerowe, zapoczątkowane
zwykle w innym celu (na przykład komunikacyjnym, rozrywkowym), które później
połączono z ARPAnet. Badacze historii Internetu wskazują na fakt, iż gwałtowny
rozwój Internetu zawdzięcza się właśnie komunikacyjnemu i rozrywkowemu aspektowi
konkurencyjnych sieci \cite{maigret2012socjologia}.

\subsection{Założenia funkcjonowania Internetu i ich realizacja}
Po tym jak w 1989 Tim Berners-Lee oraz Robert Cailliau utworzyli projekt sieci
dokumentów hipertekstowych, czyli tego, co obecnie znamy jako World Wide Web i
strony internetowe, osoby prywatne oraz instytucje komercyjne zaczęły dostrzegać
korzyści z użytkowania Internetu, a szczególnie z wykorzystania go jako medium
reklamy i sprzedaży \cite{press2015very}. Zniesienie zakazu wykorzystywania
Internetu do celów zarobkowych w 1991 roku zakończyło chwilę w której Internet
był medium naukowego dyskursu i zapoczątkowało okres istnienia Internetu dla
mas, który trwa do dziś.

\subsubsection{Perspektywa techniczna}
Podstawą struktury obecnego Internetu jest model TCP/IP i koncepcyjnie składa
się ze współpracujących ze sobą 4 warstw \cite{kahn1974protocol}:
\begin{enumerate}
	\item dostępu do sieci
	\item kontroli transportu
	\item Internetu
	\item aplikacji
\end{enumerate}
W najwyższej z warstw, czyli warstwie aplikacji działają takie usługi jak
przeglądarka czy serwer WWW. To najbardziej interesująca warstwa z perspektywy
niniejszej pracy, ale \emph{fingerprinting} urządzeń podłączonych do Internetu
może odbywać się także w niższych warstwach. % TODO Elaborate?

\subsection{Początki śledzenia użytkowników Internetu}
Gwałtowny rozwój komercyjnego Internetu sprawił, że firmy zajmujące się reklamą
i sprzedażą w Internecie zaczęły także dostrzegać korzyści płynące z
identyfikacji i śledzenia użytkowników Internetu. W szczególności zaczęto
analizować aktywność i zachowanie użytkowników. Oprócz instytucji komercyjnych,
identyfikacją i śledzeniem użytkowników zainteresowane są instytucje rządowe, co
dobitnie pokazał wyciek poufnych, tajnych i ściśle tajnych dokumentów NSA w 2013
roku. Metody identyfikacji, a zarazem śledzenia użytkowników zmieniały się w
czasie, wraz z rozwojem Internetu.

\subsubsection{Adres IPv4}
Adres IPv4 na początku istnienia Internetu był swego rodzaju globalnym
identyfikatorem, dzięki któremu można było unikatowo identyfikować użytkowników
Internetu. Adres IPv4 to \(32\) bitowy identyfikator. Prosta estymacja pozwala
nam zauważyć, że adresów IP w wersji czwartej jest około \(4,3\)
miliarda\footnote{W rzeczywistości liczba adresów IP w wersji czwartej jest
niższa.}. Internet dzisiaj, to wielomiliardowa społeczność, a liczba urządzeń
podłączonych do Internetu zdecydowanie przewyższa wyżej wymienioną estymację.
Już w 1992 roku zauważono, że w najbliższym czasie pula adresów IPv4 zostanie
wyczerpana \cite{fuller1992supernetting}. W następnych latach proponowano
kolejne rozwiązania (takie jak na przykład NAT \cite{egevang1994ip}), które
implementowali dostawcy usług internetowych, pozwalając na łączenie się wielu
urządzeń za pośrednictwem jednego, publicznego adresu IPv4. Wyczerpywanie się
kolejnych pul adresów pokazuje Rys. 1.

\begin{figure}
	\includegraphics[width=\textwidth,keepaspectratio]{img/01}
	\source{https://upload.wikimedia.org/wikipedia/commons/c/cf/Ipv4-exhaust.svg}
	\caption{Wolne pule adresów IPv4 w czasie}
\end{figure}

Biorąc pod uwagę powyższe, na mocy Zasady Szufladkowej Dirichleta możemy
stwierdzić, że adres IPv4 nie jest już identyfikatorem, który mógłby unikatowo
identyfikować każde urządzenie podłączone do Internetu. W tym momencie warto
także zaznaczyć, że o ile nowy standard IPv6 pozwalałby na taką identyfikację,
to został on zaprojektowany z myślą o prywatności i posiada szereg rozszerzeń,
które w przyszłości (kiedy Internet w pełni przejdzie na adresację w wersji
szóstej) mają zapobiegać precyzyjnej identyfikacji \cite{narten2001privacy}.

\subsubsection{Cookies}
Małe porcje informacji zapisywane na urządzeniu użytkownika, w obszarze pamięci
trwałej przeglądarki, po interakcji ze stroną internetową, która je zapisuje.
Powstały głównie ze względu na potrzebę poprawienia doświadczeń użytkowników ze
stronami internetowymi, tak aby zapamiętywać pewien stan, o znaczeniu dla danej
sesji dla danego użytkownika (na przykład stan koszyka w sklepie internetowym).

\emph{Cookies} dzielą się na tak zwane \emph{first-party cookies} i
\emph{third-party cookies}. O ile pierwsze z wymienionego podziału faktycznie
desygnowane są do tego aby spełniać wymienioną funkcję, to \emph{third-party
cookies} mogą być nadawane przez (na przykład) skrypty reklamowe, umieszczone na
serwującej je stronie, dzięki czemu użytkownik może być śledzony w kontekście
całej sieci reklamowej. Ubogie mechanizmy kontroli \emph{cookies} w
przeglądarkach internetowych i obawy związane z naruszaniem prywatności
użytkowników przez śledzenie wykorzystujące \emph{cookies}, doprowadziły do
powstania dyrektywy Unii Europejskiej dotyczącej obowiązku informacyjnego, która
zawiera m.in. obowiązek informowania o polityce stosowania \emph{cookies}.
Doprowadziło to wcześniej także do powstania rozszerzeń w przeglądarkach, takich
jak nagłówek Do Not Track i Tryb Prywatny, które w domyśle miały pomóc częściowo
rozwiązać wspomniany problem.

Niektóre przeglądarki internetowe, takie jak Apple Safari (Intelligent Tracking
Prevention silnika WebKit) lub Mozilla Firefox wykorzystują obecnie natywne
mechanizmy inteligentnego blokowania \emph{third-party cookies}. Powstało także
wiele rozszerzeń do przeglądarek, które pozwalają blokować niechciane
\emph{cookies}.

\subsubsection{Inne metody identyfikacji użytkowników}
\emph{Fingerprinting} urządzeń podłączonych do Internetu i przeglądarek
internetowych, to jedna ze zbioru wymyślnych technik, które zaczęto stosować ze
względu na ułomność i/lub postępujące ograniczenia metod wykorzystujących na
przykład adresy IP urządzeń i/lub \emph{cookies}. \emph{Fingerprinting}
przeglądarek po raz pierwszy opisał Eckersley \cite{eckersley2010unique}, jako
technikę, która pozwala serwerom WWW jednoznacznie zidentyfikować urządzenie
użytkownika za pomocą informacji wysyłanych przez przeglądarkę internetową
wtedy, kiedy te informacje są unikalne dla większości z nich, tworząc ich
\emph{fingerprint}. O relacji urządzeń i przeglądarek pomiędzy ich użytkownikami
traktuje 2.2.

\section{Fingerprinting w branży komputerowej}
\emph{Fingerprinting} to technika wykorzystywana w wielu obszarach w dyscyplinie
informatyki; \emph{fingerprinting} urządzeń podłączonych do Internetu i
przeglądarek internetowych to tylko pewien wycinek zastosowań tej koncepcji. O
ile podane na początku tego rozdziału definicje ujmują \emph{fingerprint} w
sposób ogólny i także taki, który jest spójny z definicjami, które można znaleźć
w publikacjach dotyczących \emph{fingerprintingu} urządzeń i przeglądarek, to
definicje dotyczące zastosowań \emph{fingerprintu} innych bytów mogą być
bardziej specyficzne czy też mogą eksponować inne właściwości, charakterystyczne
dla stosownych zastosowań. Kolejne punkty służą jako referencja do ukazania jak
szeroko wykorzystywana jest omawiana koncepcja.

\subsection{Fingerprinting audio, wideo i algorytmy ACR}
Metody \emph{fingerprintingu} akustycznego i cyfrowych materiałów wideo, znane
także jako technologia Automatic Content Recognition (ACR), zostały
zaprezentowane w 2012 podczas Consumer Electronics Show, pokazując, że
urządzenia dostępne dla zwykłego konsumenta mogą być na tyle sprytne by
wyszukiwać informacje w oparciu o kontekst \cite{ng2012brief}. Technologia ACR
sparametryzowana jest w podobny sposób, co algorytmy \emph{fingerprintingu}
urządzeń i przeglądarek, czyli istotny jest balans pomiędzy niepowtarzalnością a
stabilnością \emph{fingerprintu}. Klasyfikacja \emph{fingerprintu} audio lub
wideo musi działać w podobny sposób do zachowania ludzkiego moderatora, czyli w
przypadku kiedy materiał jest nieodróżnialny dla ludzkiego ucha lub oka jako
ten, wobec którego przeprowadzany jest proces rozpoznawania, to powinien on
zostać oflagowany. Proces wykorzystywany przez technologię Automatic Content
Recognition określany jest mianem \emph{perceptual hashing}.

\subsection{Fingerprinting klucza publicznego}
W kryptografii klucza publicznego, w celach autoryzacji klucza publicznego
pozyskanego w niezaufany sposób (na przykład ściągając go ze strony
internetowej), poprzez zaufany kanał wymiany informacji (zwykle rozmowa
telefoniczna), który nie pozwala na autoryzację całego klucza w efektywny
sposób, stosuje się jego skrót, nazywany ,,fingerprintem klucza publicznego''.
Wynik zastosowania odpowiedniej funkcji skrótu na kluczu publicznym jest na tyle
kompaktowy, że pozwala na efektywną manualną autoryzację, czyli ręcznie przez
człowieka.

W celu ułatwienia wymiany \emph{fingerprintów} kluczy publicznych poprzez kanały
głosowe powstała lista słów PGP, która analogicznie do alfabetu fonetycznego
NATO asocjuje każdą kolejną porcję bitów \emph{fingerprintu} klucza z
odpowiednim słowem w języku angielskim.
