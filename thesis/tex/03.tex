\chapter{Metody fingerprintingu urządzeń i przeglądarek}
Od kiedy Eckersley \cite{eckersley2010unique} po raz pierwszy opisał
\emph{fingerprinting} przeglądarek, metody \emph{fingerprintingu} zmieniały się
i ewoluowały razem z nimi. Sytuacja \emph{fingerprintingu} poszczególnych
protokołów czy implementacji stosu TCP/IP jest raczej odmienna. Choć te metody
znane są już od lat dziewięćdziesiątych (a ich implementacje to takie narzędzia
jak nmap), są relatywnie mniej ,,przebojowym'' tematem badań i eksperymentów.
Niniejszy rozdział enumeruje przez szeroką gamę metod, prezentując implementacje
najważniejszych i rozważając ich skuteczność.

\section{Pojęcie pasywnego i aktywnego fingerprintingu}
\emph{Fingerprinting} może być przeprowadzany w sposób pasywny lub aktywny.
Pasywny polega całkowicie na informacjach, do których zdobycia nie jest wymagana
interakcja z drugą stroną połączenia (np. nagłówki HTTP (takie jak
\emph{User-Agent}))---te informacje i tak zostają wysłane. Aktywny
\emph{fingerprinting} opiera się z kolei na przykład na użyciu skryptów do
wydobycia większej ilości informacji (takich jak konfiguracja przeglądarki)
\cite[s. 3]{al2020too}.

\section{Źródła danych}

\section{Fingerprinting implementacji stosu TCP/IP}

% \section{Źródła danych identyfikacyjnych urządzeń}

% \subsection{Protokoły warstwy drugiej w modelu OSI}

% \subsection{Stos TCP/IP}

% \subsubsection{Protokoły warstwy 3 i 4 w modelu OSI}

% \paragraph{IPv4}

% \paragraph{IPv6}

% \paragraph{ICMP}

% \paragraph{IEEE802.11}

% \subsection{Nierutowalne protokoły warstwy 5 (lokalny fingerprinting)}

% \subsection{Rutowalne protokoły warstwy 7}

% \section{Wybrane metody fingerprintingu urządzeń i ich systemów operacyjnych}

% \section{Przeglądarki jako specjalny przypadek fingerprintingu urządzeń}

% \section{Źródła danych identyfikacyjnych przeglądarek}

% \section{Wybrane metody fingerprintingu przeglądarek}

% \subsection{Implementacje wybranych metod fingerprintingu przeglądarek}

% \section{Implementacja przykładu identyfikacji przeglądarki}
