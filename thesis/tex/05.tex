\chapter{Podsumowanie}
Wszystko wskazuje na to, że \emph{fingerprinting} przeglądarek internetowych i
urządzeń podłączonych do Internetu nie odejdzie przez najbliższy czas w
niepamięć. Ta kontrowersyjna technika nie jest już jedynie trendem w gronie
metod śledzenia użytkowników, ale wygląda na to, że stała się integralną częścią
współczesnego Internetu. Wiele firm działających w obszarze i kształtujących
Internet (w szczególności te związane z reklamą w Internecie) nie ma w interesie
ograniczania \emph{fingerprintingu} \cite{al2018beyond}.
Najpopularniejsza\footnote{https://netmarketshare.com/browser-market-share.aspx}
przeglądarka Chrome jest najprawdopodobniej najbardziej podatną \cite{al2017not}
na tę technikę przeglądarką internetową m.in. z tego powodu. Pomimo iż
specjaliści od cyberbezpieczeństwa traktują \emph{fingerprinting} nie tylko jako
lukę w zabezpieczeniach, ale także w prywatności
\cite{mowery2012pixel,al2020too}, to natychmiastowe ograniczenie go sprawi, że
część Internetu może najzwyczajniej przestać działać. Nie tylko ze względu na
to, że polega on na podstawowych funkcjonalnościach przeglądarek i ich pewnym
rozproszeniu, ale także dlatego, że niektóre serwisy polegają na
\emph{fingerprintingu} w kontekście uwierzytelniania użytkowników (na tej
zasadzie działają takie technologie jak reCAPTCHA v3) \cite{45581}. Użycie i
ochrona przed tą techniką przez służby mundurowe także zasługuje na istotną
uwagę.

Ciągle powstają nowe metody \emph{fingerprintingu}, a te opisane w pracy
stanowią jedynie fundamentalne podstawy \emph{fingerprintingu} przeglądarek i
urządzeń. Są one jednak nadal skuteczne a tam, gdzie kończą się ich
,,naturalne'' możliwości identyfikacyjne, mogą pomóc klasyfikatory takie jak
eksperymentalny klasyfikator opisany w czwartym rozdziale.

Idea stojąca za \emph{fingerprintingiem} jest relatywnie prosta. Mimo tego
manualne stosowanie tej idei w odniesieniu do \emph{fingerprintingu}
przeglądarek i urządzeń jest nietrywialne. Wymaga ono dobrej znajomości
ekosystemu przeglądarek, systemów operacyjnych, zależności pomiędzy różnymi
parametrami tych ekosystemów itd. Można zatem dojść do wniosku, że tworzenie
gotowych produktów jak biblioteki \emph{fingerprintingowe} (np. wspomniany
\textbf{fingerprintjs}\footnote{https://github.com/fingerprintjs/fingerprintjs2})
lub zaprezentowany eksperymentalny klasyfikator przyczynią się do skorzystania z
omawianych metod większej grupy zainteresowanych. To i opisane wcześniej
problemy są pewnymi wyzwaniami w walce o bezpieczeństwo i prywatność w
Internecie. Rozwiązania tych problemów są niestety często bardziej skomplikowane
od trudności w użyciu metod, których istnienie generuje te problemy. Powyższe
aspekty składają się na wniosek, że \emph{fingerprinting} przeglądarek
internetowych i urządzeń podłączonych do Internetu jest tematem, który warto i
należy dalej zgłębiać---nie tylko przez osoby zajmujące się
cyberbezpieczeństwem.
