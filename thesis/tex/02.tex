\chapter{Problematyka prywatności i anonimowości w Internecie}
Technika śledzenia urządzeń, przeglądarek i użytkowników o której traktuje ta
praca jest ,,chodliwym'' tematem głównie dlatego, że jej implikacje w obszarze
prywatności i anonimowości są istotne.

\section{Prywatność i anonimowość}
Prywatność internetowa to pewien podzbiór relacji pomiędzy gromadzeniem i
rozpowszechnianiem danych, technologią, społecznym oczekiwaniom wobec
prywatności i prawnymi oraz politycznym problemami orbitującymi wokół tych
zagadnień \cite{michael2014uberveillance}. W celach orientacyjnych, stosowanym
uproszczeniem jest pogląd, że prywatność internetowa to możliwość zatrzymania
pewnych (związanych z własną aktywnością) danych dla siebie. Anonimowością jest
zatem sytuacja w której przekazywanie takich danych jest zablokowane.

Prywatność i anonimowość użytkowników Internetu stała się zagadnieniem jeszcze
przed nadejściem ery Internetu \cite{david1965some}, pozwalając na koniec lat
dziewięćdziesiątych na zaognienie dyskusji, która trwa do dzisiaj. Użytkownicy
Internetu mają różne oczekiwania wobec poziomu ich prywatności w sieci. Mniej
wyczuleni na punkcie prywatności użytkownicy są w stanie pójść na pewny
kompromis pomiędzy wykorzystaniem ich danych a oferowanym na tej podstawie,
potencjalnym usprawnieniem ich doświadczeń w Internecie (np. na konkretnych
stronach internetowych, w kontekście sieci reklamowych lub w innych, szerszych
kontekstach). Akceptują oni ryzyko zbyt szczegółowego profilowania,
potencjalnych naruszeń prywatności i inwigilacji. Inni użytkownicy dążą do
utrzymania anonimowości, takiej jaka panowała w Internecie na początku jego
istnienia \cite{snowden2019pamiec}[s. 54--69].

\subsection{Łączenie danych}
Istnieją firmy specjalizujące się w pozyskiwaniu, kupowaniu i przetwarzaniu
danych użytkowników w celach reklamowych lub w innych celach. Nie wszystkie
firmy rynku danych przetwarzają dane w sposób naruszający prywatność
użytkowników, ale techniki takie jak łączenie danych z różnych źródeł (bez
uprzedniej anonimizacji) mogą lub będą prowadzić do nadużyć.

Bardzo sławny przypadek firmy Cambridge Analytica, która tworzyła profile
psychologiczne użytkowników i używała ich do manipulacji opinią publiczną za
pomocą mediów społecznościowych, to jeden z przykładów firmy, która łącząc dane
poważnie naruszyła prywatność użytkowników (m.in. Facebooka).

Istnieje wiele różnych dróg dzięki którym w Internecie i w świecie rzeczywistym
użytkownicy będą nieświadomie profilowani i/lub inwigilowani, a łączenie danych
z różnych źródeł istotnie poprawi dokładność obrazu użytkownika. Jedną z takich
dróg jest używanie stron internetowych, będących częścią większej sieci
reklamowej, która łączy dane, na przykład z portali ,,social media'', wyników
wyszukiwań, wysyłanych formularzy, a nawet z takich źródeł jak systemów
automatycznego wykrywania twarzy w sklepach stacjonarnych. Niektóre informacje,
które można wnioskować po automatycznym przetworzeniu, to orientacja seksualna,
poglądy polityczne i religijne, rasa, użycie substancji psychoaktywnych,
estymowany iloraz inteligencji czy osobowość \cite{kosinski2013private}.

\subsection{Argument pt. ,,nic do ukrycia''}

\section{Porównanie metod śledzenia użytkowników}

\section{Zagrożenia związane z fingerprintingiem}

\section{Możliwości ochrony prywatności}

\section{Pozytywne aspekty fingerprintingu}
