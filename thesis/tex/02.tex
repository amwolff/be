\chapter{Problematyka prywatności i anonimowości w Internecie}
Technika, o której traktuje ta praca, jest ważnym tematem głównie dlatego, że
wykorzystanie jej do identyfikacji użytkowników ma istotne implikacje w obszarze
prywatności i anonimowości.

\section{Prywatność i anonimowość}
Prywatność internetowa to pewien podzbiór relacji pomiędzy gromadzeniem i
rozpowszechnianiem danych, technologią, społecznym oczekiwaniom wobec
prywatności i prawnymi oraz politycznym problemami orbitującymi wokół tych
zagadnień \cite{michael2014uberveillance}. W celach orientacyjnych stosowanym
uproszczeniem jest pogląd, że prywatność internetowa to możliwość zatrzymania
związanych z własną aktywnością danych dla siebie. Anonimowością jest zatem
sytuacja, w której przekazywanie takich danych jest zablokowane.

Prywatność i anonimowość użytkowników Internetu stała się zagadnieniem jeszcze
przed nadejściem ery Internetu \cite{david1965some}, pozwalając na koniec lat
dziewięćdziesiątych na zaognienie dyskusji, która trwa do dzisiaj. Użytkownicy
Internetu mają różne oczekiwania wobec poziomu ich prywatności w sieci. Mniej
wyczuleni na punkcie prywatności użytkownicy są w stanie pójść na pewny
kompromis pomiędzy wykorzystaniem ich danych a oferowanym na tej podstawie
potencjalnym usprawnieniem ich doświadczeń w Internecie (na przykład na
konkretnych stronach internetowych, w kontekście sieci reklamowych lub w innych
szerszych, kontekstach). Akceptują oni ryzyko zbyt szczegółowego profilowania,
potencjalnych naruszeń prywatności i inwigilacji. Inni użytkownicy dążą (mniej
lub bardziej) do utrzymania anonimowości takiej, jaka panowała w Internecie na
początku jego istnienia \cite[s. 54--69]{snowden2019pamiec}.

\subsection{Łączenie danych}
Istnieją firmy specjalizujące się w pozyskiwaniu, kupowaniu i przetwarzaniu
danych użytkowników w różnych celach (zwykle reklamowych). Nie wszystkie firmy
rynku danych przetwarzają dane w sposób naruszający prywatność użytkowników, ale
techniki takie jak łączenie danych z różnych źródeł (bez uprzedniej
anonimizacji) mogą lub będą prowadzić do nadużyć.

Bardzo sławny przypadek firmy Cambridge Analytica, która tworzyła profile
psychologiczne użytkowników i używała ich do manipulacji opinią publiczną za
pomocą mediów społecznościowych to jeden z przykładów firmy, która łącząc dane,
poważnie naruszyła prywatność użytkowników (m.in. Facebooka).

Istnieje wiele różnych dróg, dzięki którym w Internecie i w świecie rzeczywistym
użytkownicy będą nieświadomie profilowani lub inwigilowani, a łączenie danych z
różnych źródeł istotnie poprawi dokładność obrazu użytkownika. Jedną z takich
dróg jest używanie stron internetowych będących częścią większej sieci
reklamowej, która łączy dane na przykład z portali \emph{social media}, wyników
wyszukiwań, wysyłanych formularzy, a nawet z takich źródeł jak systemów
automatycznego wykrywania twarzy w sklepach stacjonarnych. Niektóre informacje,
które można wnioskować po automatycznym przetworzeniu to orientacja seksualna,
poglądy polityczne i religijne, rasa, historia użycia substancji
psychoaktywnych, estymowany iloraz inteligencji czy osobowość
\cite{kosinski2013private}.

\subsection{Czy prywatność jest nam potrzebna?}
Ochrona prywatności ma swoich przeciwników i zwolenników. ,,nie obchodzi mnie
prywatność, bo nie mam nic do ukrycia'' to argument przytaczany przez
przeciwników ochrony prywatności. Jedną z ważnych osób, które opowiedziały się
niegdyś za taką argumentacją, jest Eric Schmidt, były CEO firmy Google. Istnieje
silna polaryzacja pomiędzy przeciwnikami i zwolennikami ochrony prywatności. W
swojej książce autobiograficznej Edward Snowden (słynny amerykański
\emph{whistleblower}) stwierdził, że ,,oświadczyć, że nie obchodzi cię
prywatność, bo nie masz nic do ukrycia, to mniej więcej to samo, co oświadczyć,
że nie obchodzi cię wolność słowa, ponieważ nie masz nic do powiedzenia''
\cite{snowden2019pamiec}. Bruce Schneier, amerykański kryptograf i ekspert w
dziedzinie bezpieczeństwa komputerowego podsumowuje, że zbyt wiele osób myśli o
tym argumencie jak o wyborze pomiędzy bezpieczeństwem a prywatnością.
,,Prawdziwym wyborem jest wybór pomiędzy wolnością a kontrolą''
\cite{schneier2006eternal}. Prawo do prywatności zawiera się w Powszechnej
Deklaracji Praw
Człowieka\footnote{https://www.ohchr.org/EN/UDHR/Pages/Language.aspx?LangID=pql}.

\section{Porównanie metod identyfikacji użytkowników}
Tak ja zauważono wcześniej w niniejszej pracy---wraz z komercjalizacją Internetu
powstał i ewoluował szereg różnych metod identyfikacji urządzeń, przeglądarek i
tym samym użytkowników. Istnieją różne zastosowania identyfikacji, ale jednym z
najbardziej powszechnie omawianych (i kontrowersyjnych) jest śledzenie
użytkowników (formalnie: łączenie ze sobą wielu wizyt jednego użytkownika na tej
samej platformie) \cite[s. 3]{al2020too}.

Warto także zauważyć, że \emph{fingerprinting} przeglądarek internetowych to
jedna z dróg identyfikacji urządzeń podłączonych do Internetu. Jest ona jednak
na tyle reprezentatywna, że często mówi się o \emph{fingerprintingu} urządzeń
jako \emph{fingerprintingu} przeglądarek. Dzieje się tak, ponieważ dzisiejsze
przeglądarki internetowe, podczas interakcji z serwerami WWW, mogą aktywnie
i/lub pasywnie przekazywać zestaw danych na tyle szeroki
\cite{eckersley2010unique}, że zawiera on w sobie \emph{fingerprint} urządzenia
na którym działa przeglądarka. Przeważający ogrom aktywności użytkowników wokół
,,przeglądania'' Internetu i szeroki wachlarz danych ,,oferowanych'' przez
przeglądarki internetowe naturalnie sprawia, że wysiłki identyfikujące
użytkowników rozważane są głównie w kontekście tychże. Niniejsza praca stara się
zapewnić pewien (choć często niewidoczny) podział. W kolejnym rozdziale
niniejszej pracy omówiony jest także \emph{fingerprinting} urządzeń podłączonych
do Internetu, kiedy niemożliwe jest wykorzystanie do celów identyfikacyjnych
przeglądarki internetowej użytkownika.

Identyfikacja użytkowników nie zawsze jest także synonimem z identyfikacją
urządzeń czy przeglądarek, ale odsetek przypadków, kiedy w dzisiejszym świecie
więcej niż jedna osoba korzysta z np. jednej przeglądarki internetowej, wydaje
się być stosunkowo niski. Powodem dla takiej estymacji jest fakt, iż wspomniane
przeglądarki, te które posiadają istotny ułamek udziałów w rynku przeglądarek
internetowych, posiadają mechanizmy pozwalające na ich jednoznaczną
personalizację (parowanie z personalnym kontem Google w przeglądarce Google
Chrome, integracja przeglądarki Safari z ekosystemem firmy Apple, parowanie z
personalnym kontem Firefox w przeglądarce Firefox itd.). Co więcej, jeden
użytkownik może korzystać z wielu przeglądarek, co utrudnia jednoznaczną
identyfikację, ale nie czyni jej niemożliwą. Istnieją metody
\emph{fingerprintingu} urządzeń i przeglądarek pozwalające na identyfikację
użytkowników pomiędzy przeglądarkami internetowymi, zainstalowanymi na tym samym
urządzeniu.

Gdyby podsumować metody identyfikacji użytkowników w dzisiejszym Internecie, to
lista takich technik prezentowałaby się następująco:

\section{Zagrożenia związane z fingerprintingiem}

\section{Możliwości ochrony prywatności}

\section{Pozytywne aspekty fingerprintingu}
