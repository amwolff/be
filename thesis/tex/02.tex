\chapter{Problematyka prywatności i anonimowości w Internecie}
Technika o której traktuje ta praca jest ważnym tematem głównie dlatego, że jej
wykorzystanie do identyfikacji urządzeń, przeglądarek i użytkowników ma istotne
implikacje w obszarze prywatności i anonimowości.

\section{Prywatność i anonimowość}
Prywatność internetowa to pewien podzbiór relacji pomiędzy gromadzeniem i
rozpowszechnianiem danych, technologią, społecznym oczekiwaniom wobec
prywatności i prawnymi oraz politycznym problemami orbitującymi wokół tych
zagadnień \cite{michael2014uberveillance}. W celach orientacyjnych, stosowanym
uproszczeniem jest pogląd, że prywatność internetowa to możliwość zatrzymania
pewnych (związanych z własną aktywnością) danych dla siebie. Anonimowością jest
zatem sytuacja w której przekazywanie takich danych jest zablokowane.

Prywatność i anonimowość użytkowników Internetu stała się zagadnieniem jeszcze
przed nadejściem ery Internetu \cite{david1965some}, pozwalając na koniec lat
dziewięćdziesiątych na zaognienie dyskusji, która trwa do dzisiaj. Użytkownicy
Internetu mają różne oczekiwania wobec poziomu ich prywatności w sieci. Mniej
wyczuleni na punkcie prywatności użytkownicy są w stanie pójść na pewny
kompromis pomiędzy wykorzystaniem ich danych a oferowanym na tej podstawie,
potencjalnym usprawnieniem ich doświadczeń w Internecie (np. na konkretnych
stronach internetowych, w kontekście sieci reklamowych lub w innych, szerszych
kontekstach). Akceptują oni ryzyko zbyt szczegółowego profilowania,
potencjalnych naruszeń prywatności i inwigilacji. Inni użytkownicy dążą (mniej
lub bardziej) do utrzymania anonimowości, takiej jaka panowała w Internecie na
początku jego istnienia \cite[s. 54--69]{snowden2019pamiec}.

\subsection{Łączenie danych}
Istnieją firmy specjalizujące się w pozyskiwaniu, kupowaniu i przetwarzaniu
danych użytkowników w celach reklamowych lub w innych celach. Nie wszystkie
firmy rynku danych przetwarzają dane w sposób naruszający prywatność
użytkowników, ale techniki takie jak łączenie danych z różnych źródeł (bez
uprzedniej anonimizacji) mogą lub będą prowadzić do nadużyć.

Bardzo sławny przypadek firmy Cambridge Analytica, która tworzyła profile
psychologiczne użytkowników i używała ich do manipulacji opinią publiczną za
pomocą mediów społecznościowych, to jeden z przykładów firmy, która łącząc dane
poważnie naruszyła prywatność użytkowników (m.in. Facebooka).

Istnieje wiele różnych dróg dzięki którym w Internecie i w świecie rzeczywistym
użytkownicy będą nieświadomie profilowani i/lub inwigilowani, a łączenie danych
z różnych źródeł istotnie poprawi dokładność obrazu użytkownika. Jedną z takich
dróg jest używanie stron internetowych, będących częścią większej sieci
reklamowej, która łączy dane, na przykład z portali ,,social media'', wyników
wyszukiwań, wysyłanych formularzy, a nawet z takich źródeł jak systemów
automatycznego wykrywania twarzy w sklepach stacjonarnych. Niektóre informacje,
które można wnioskować po automatycznym przetworzeniu, to orientacja seksualna,
poglądy polityczne i religijne, rasa, użycie substancji psychoaktywnych,
estymowany iloraz inteligencji czy osobowość \cite{kosinski2013private}.

\subsection{Czy prywatność jest nam potrzebna?}
Jednym z argumentów stosowanych przez przeciwników ochrony prywatności jest
sformułowanie ,,nie obchodzi mnie prywatność, bo nie mam nic do ukrycia''.
Zwolennikiem takiej argumentacji jest między innymi Eric Schmidt, były CEO firmy
Google. Argument pt. ,,nic do ukrycia'' ma także rzeszę przeciwników, głównie
osób zaangażowanych w ochronę prywatności. W swojej książce autobiograficznej
Edward Snowden (słynny amerykański \emph{whistleblower}) stwierdził, że
,,oświadczyć, że nie obchodzi cię prywatność, bo nie masz nic do ukrycia, to
mniej więcej to samo, co oświadczyć, że nie obchodzi cię wolność słowa, ponieważ
nie masz nic do powiedzenia.'' \cite{snowden2019pamiec}. Bruce Schneier,
amerykański kryptograf i ekspert w dziedzinie bezpieczeństwa komputerowego,
podsumowuje, że zbyt wiele osób myśli o tym argumencie jak o wyborze pomiędzy
bezpieczeństwem a prywatnością. ,,Prawdziwym wyborem jest wybór pomiędzy
wolnością a kontrolą'' \cite{schneier2006eternal}.

\section{Porównanie metod śledzenia użytkowników}
Tak ja zauważono wcześniej w niniejszej pracy---wraz z komercjalizacją Internetu
powstał i ewoluował szereg różnych metod śledzenia urządzeń, przeglądarek i
użytkowników. Śledzenie użytkowników nie jest synonimem ze śledzeniem urządzeń
czy przeglądarek, ale odsetek przypadków w których w dzisiejszym świecie więcej
niż jedna osoba korzysta z np. jednej przeglądarki wydaje się być stosunkowo
niski. Powodem dla takiej estymacji jest fakt, iż wspomniane przeglądarki, te
które posiadają istotny odsetek udziałów w rynku przeglądarek, posiadają
mechanizmy pozwalające na ich jednoznaczną personalizację (parowanie z
personalnym kontem Google w przeglądarce Google Chrome, integracja przeglądarki
Safari z ekosystemem firmy Apple, parowanie z personalnym kontem Firefox w
przeglądarce Firefox itd.).

Gdyby podsumować metody śledzenia użytkowników w dzisiejszym Internecie, to
lista takich technik prezentowałaby się następująco:

\section{Zagrożenia związane z fingerprintingiem}

\section{Możliwości ochrony prywatności}

\section{Pozytywne aspekty fingerprintingu}
