\chapter{Eksperymentalny algorytm klasyfikatora fingerprintów}

\section{Motywacja}
Wcześniej w niniejszej pracy zostało zaznaczone, że \emph{fingerprint} może być
bardziej unikalny kosztem jego stabilności i odwrotnie. Aby zgrupować kolejne
\emph{fingerprinty} tej samej przeglądarki internetowej, których różnice
wynikają z naturalnych przemian tej przeglądarki (np. aktualizacja), możemy użyć
klasyfikatora opartego na odpowiednim algorytmie. Tę relację przedstawia Rys. 7.
Dzięki takiemu rozwiązaniu jesteśmy w stanie zachować wiarygodny wskaźnik
entropii, używając średnio bardziej unikalnych \emph{fingerprintów} o mniejszej
stabilności.

\begin{figure}
	\includegraphics[width=\textwidth,keepaspectratio]{img/09}
	\source{Własne}
	\caption{Relacja \emph{fingerprint}-przeglądarka}
\end{figure}

\section{Opis algorytmu}

\section{Ocena złożoności czasowej i pamięciowej}

\section{Ocena efektywności}
