\chapter{Wprowadzenie do fingerprintingu}

\section{Podstawowe pojęcia}

\subsection{Nomenklatura używana w tej pracy}
Pisząc o odcisku palca użyto (także w tytule pracy) ogólnie przyjętego skrótu
myślowego, oznaczającego odbitkę linii papilarnych, czyli formę językową
uznawaną za poprawną przez specjalistów od daktyloskopii.

Użycie formy językowej ,,odcisk palca'' w terminie ,,cyfrowy odcisk palca'' ma
wiele sensu. Jeszcze bez zdefiniowania tego specjalistycznego terminu, możemy
domyślić się, co oznacza. Oczywiście wynika to z faktu, że cyfrowy odcisk palca
i analogowy odcisk palca są ze sobą w pewien sposób powiązane (koncepcja
cyfrowego odcisku palca czerpie z wartości wynikających ze stosowania odbitek
ludzkich linii papilarnych w dziedzinie kryminalistyki).

Angielskie słowo ,,fingerprint'' tłumaczy się jako odcisk palca, jednakże w
zagranicznych publikacjach dotyczących cyfrowego odcisku palca rzadko występuje
termin ,,digital fingerprint''. Kontekst użycia jest na tyle wyraźny, że użycie
samego ,,fingerprint'' jest wystarczające.

Zachodnie nazewnictwo ma tę przewagę, że jest zdecydowanie bardziej kompaktowe.
Także w przypadku słowotwórczego zabiegu \emph{fingerprinting}, oznaczającego
czynność; szukając polskiego odpowienika musielibyśmy sięgnąć po ,,cyfrowe
znakowanie''. Z uwagi na tę kompaktowość i łatwość użycia, w pracy preferowane
będzie użycie oryginalnej nomenklatury.

\subsection{Definicje}
W kolejnych punktach zawarto najważniejsze definicje i powiązane pojęcia, które
będą używane w przeciągu całej pracy.

\subsubsection{Fingerprint}
Wektor cech pozwalający zidentyfikować dowolny zbiór danych.

Aby fingerprint pełnił praktyczną funkcję identyfikacyjną, tak jak odbitka
ludzkich linii papilarnych pełni praktyczną funkcję identyfikacyjną, często
stosuje się algorytm, który kojarzy wektor cech z określonej długości (zwykle
krótkim) ciągiem bajtów (identyfikatorem). Takim algorytmem może być na przykład
wysokiej wydajności funkcja skrótu (niekoniecznie zdatna do zastosowań
kryptograficznych---na przykład MurmurHash). W niektórych źródłach można także
spotkać się z taką definicją, że fingerprint to już sam wynik wyżej wspomnianego
algorytmu. Taka definicja nie zmienia istoty fingerprintu, ale jest mniej
przydatna w kontekście fingerprintingu urządzeń podłączonych do Internetu i
przeglądarek internetowych, czego dotyczy niniejsza praca.

\subsubsection{Fingerprint urządzenia podłączonego do Internetu}
Wektor cech pozwalający zidentyfikować urządzenie podłączone do Internetu.

\subsubsection{Instalacja przeglądarki internetowej}
Instalacja na konkretnym urządzeniu. W przypadku zmiany ustawień, konfiguracji i
liczby pluginów oraz aktualizacji przeglądarki, instalacja przeglądarki
pozostaje ciągle tą samą instalacją.

\subsubsection{Fingerprint przeglądarki internetowej}
Wektor cech pozwalający zidentyfikować instalację przeglądarki internetowej.

\subsection{Właściwości fingerprintu}
Ludzkie linie papilarne są na ogół niepowtarzalne, niezmienne i nieusuwalne. Z
wartości wynikających ze stosowania ich w swojej dziedzinie badawczej czerpie
(także etymologicznie) koncepcja fingerprintu i dlatego też fingerprint z dobrze
dobranymi cechami będzie odzwierciedlać podobne właściwości.

W przypadku fingerprintingu urządzeń podłączonych do Internetu i przeglądarek
internetowych najważniejszymi ich właściwościami są unikalność / różnorodność
(niepowtarzalność) oraz stabilność (niezmienność), przy czym zwiększenie
unikalności lub stabilności ma najczęściej negatywny wpływ na drugi parametr.

Jedną ze stosowanych\footnote{Metrykę tę stosowało na przykład badanie ,,How
	Unique Is Your Web Browser?'' Electronic Frontier Foundation; w momencie
pisania pracy jedno z największych badań tego typu.} metod pomiaru unikalności
fingerprintu urządzeń i przeglądarek jest entropia Shannona.

\subsubsection{Entropia Shannona}
Wartość entropii można rozumieć jako liczbę pytań binarnych potrzebnych do
sklasyfikowania losowo wybranego elementu z danego zbioru. Entropia Shannona
zbioru \(D\) z etykietami \(\{l_{0}, l_{1}, \dots, l_{n - 1}\}\) wyraża się
wzorem \[H(D) = -{\sum_{i = 0}^{n - 1}{p(l_{i})\log_{2}{p(l_{i})}}}\] gdzie
\(p(l_{i})\) to wyrażona ułamkiem częstość \(x \in D\) mającego etykietę
\(l_{i}\). W przypadku w którym każda etykieta występuje tak samo często
entropia ma wartość maksymalną równą \(\log_{2}{(n - 1)}\).

Przykład: jeśli zbiór fingerprintów przeglądarek internetowych ma \(32\) bity
entropii, to w przypadku losowego wyboru jednego z nich oczekujemy, że w
najlepszym przypadku tylko \(1\) na \(4294967295\) przeglądarek będzie miała
taki sam fingerprint.

\section{Fingerprinting a Internet} % https://sjp.pwn.pl/poradnia/haslo/;228
Aby lepiej zrozumieć istotę fingerprintu i motywację stojącą za stosowaniem
fingerprintingu w różnych obszarach techniki, kolejne punkty posłużą jako
referencja (także historyczna).

\subsection{Podstawy funkcjonowania Internetu}

\subsection{Założenia funkcjonowania Internetu}

\subsection{Realizacja założeń funkcjonowania Internetu}
