\chapter*{Wstęp}
\addcontentsline{toc}{chapter}{Wstęp}
Temat niniejszej pracy został wybrany głównie ze względu na chęć przedstawienia
niektórych aspektów metod cyfrowego odcisku palca przeglądarek internetowych
i~urządzeń podłączonych do Internetu społeczności akademickiej w~Polsce:
możliwościach, zagrożeniach, wyzwaniach i~rozwiązaniach z~nim związanych. Autor
niniejszej pracy zauważa także powiększający się problem kryzysu prywatności
w~Internecie, który często przenika także do świata fizycznego.

Internet stał się nieco niebezpiecznym miejscem dla swoich wirtualnych
obywateli. Dziś muszą oni uważać na takie niebezpieczeństwa jak złośliwe
oprogramowanie, ale także ich prywatność jest zagrożona przez cały czas.
Identyfikacja użytkowników Internetu zaczęła być pożądana w~chwili kiedy
Internet stał się częścią tzw. mass media. Platformy reklamowe zwykle
wykorzystują możliwość identyfikacji użytkowników do śledzenia ich w~sieci, ale
zostało pokazane, że często można ich także skutecznie zdeanonimizować. W~tym
kontekście tworzenie cyfrowych odcisków palca (\emph{fingerprinting}) to jedna
z~metod identyfikacji, która pozwala serwerom WWW na jednoznaczną identyfikację
urządzeń użytkowników za pomocą informacji wysyłanych przez to urządzenie lub
przeglądarkę internetową (działającą na tym urządzeniu), wtedy kiedy te
informacje są unikalne dla większości z~nich, tworząc ich cyfrowy odcisk palca
(\emph{fingerprint}). Pisząc bardziej obrazowo: w~przeciwieństwie do
\emph{cookies} i~lokalnych obiektów przechowywania cyfrowy odcisk przeglądarki
pozostaje ten sam w~tzw. trybie prywatnym (incognito), a~nawet gdy użytkownik
usunie wszystkie dane przeglądarki internetowej lub zresetuje ją do ustawień
fabrycznych. Jest to zatem identyfikacja niezależna od stanu, którą niezwykle
ciężko wykryć (jak się okazuje). Biorąc pod uwagę te właściwości, jest to
zupełna nowość w~tej materii. Tworzenie cyfrowych odcisków rzutuje m.in. na
obszary bezpieczeństwa komputerowego i~prywatności. W~historii Internetu
bezpieczeństwo i~prywatność były często rozważanymi tematami; dyskusja
o~bezpośrednich zagrożeniach w~związku z~tymi obszarami i~czynne zapobieganie
ich rozwojowi jest zwykle sprawą najwyższej wagi.

Niniejsza praca ma na celu zapoznanie odbiorców z~koncepcją cyfrowych odcisków,
zbadanie problematyki prywatności i~anonimowości w~ich kontekście i~wyróżnienie
oraz szczegółowe opisanie kluczowych metod tej techniki identyfikacji
użytkowników. Oprócz analizy aktywnie wykorzystywanych metod praca koncentruje
się na genezie cyfrowych odcisków, zagrożeniach, jakie ze sobą niosą
i~pozytywnych aspektach dostępności kolejnej możliwości identyfikacji
użytkowników.

Rozważana jest również skuteczność omawianych metod i~możliwości jej
zwiększenia. Ponieważ niektóre odciski mogą zmieniać się na przykład wraz ze
zmianą wersji przeglądarki (np. poprzez jej aktualizację), proponowane
ulepszenie polega na utworzeniu klasyfikatora do grupowania podobnych do siebie
odcisków. W~pracy przedstawiono heurystyczny algorytm, który wykorzystuje
odległość Levenshteina do obliczania kumulatywnej różnicy między tekstowymi
komponentami dwóch odcisków.

Ze względów logicznych i~objętościowych w~pracy pominięto nieużywane już (zwykle
dotyczące identyfikacji urządzeń, które nie komunikują się za pomocą
przeglądarki) lub bardzo szczegółowe metody, których dokładna implementacja
i~działanie jest silnie zależne od konkretnych potrzeb.

Stan wiedzy badanego obszaru jest relatywnie słaby, ale wynika to z~jego
młodości. O~tworzeniu cyfrowych odcisków przeglądarek internetowych i~urządzeń
podłączonych do Internetu rozmawia się od około dekady. Wtedy opublikowana
zostaje pierwsza praca opisująca niektóre z~podstawowych efektów, z~którymi mamy
do czynienia do tej pory. Większość badań powstała ze względu na obserwację
rozwoju opisywanych metod ,,w środowisku naturalnym''; tworzenie cyfrowych
odcisków używane było przez różne firmy przed zgłębianiem tematu w~środowisku
akademickim. Obszar jest zatem dosyć młody, ale liczba publikacji z~nim
związanych nie przestaje rosnąć.

Problemy, które wynikły w~trakcie pisania pracy to brak cieszącej się
autorytetem literatury i~niewielka liczba nowych publikacji poruszających pewne
szczególne aspekty omawianego obszaru. Wynika to najpewniej ze zbyt szybko
zmieniających się środowisk, które należałoby opisywać. Bardzo wartościowe
publikacje, na które autor chciałby zwrócić uwagę ze względu na ich wysoką
jakość to ostatnie prace dotyczące \emph{fingerprintingu} autorstwa CJ
Mitchella, W~Li i~NM Al-Fannaha z~Uniwersytetu Londyńskiego. Pewnym problemem
było także przeprowadzanie testów heurystycznych algorytmów klasyfikatora ze
względu na potrzebę uczestnictwa w~nich sporej liczby urządzeń/użytkowników.
Dzięki pomocy rodziny i~znajomych autora ten problem był mniej dotkliwy.

Praca podzielona jest na cztery rozdziały. W~pierwszym rozdziale opisano jak
rozumieć i~zdefiniowano niekonwencjonalne pojęcia, skróty używane w~pracy.
Opisano historię, podstawy funkcjonowania Internetu (założenia i~ich realizację)
i~szczegółowo tło, na jakim zarysowuje się nowa, opisywana technika
identyfikacji użytkowników. W~drugim rozdziale zbadano problematykę prywatności
i~anonimowości w~Internecie oraz wpływ na nią opisywanej techniki. Rozdział ten
zawiera także przegląd wyzwań i~rozwiązań dotyczących tworzenia cyfrowych
odcisków, a~także ich potencjał w~innych, mniej oczywistych obszarach. Trzeci
rozdział zawiera przegląd wybranych metod i~autorskie przykłady prezentujące
możliwości tychże. Czwarty rozdział jest najbardziej eksperymentalnym, ale także
,,produktowym'' rozdziałem. W~tym rozdziale przedstawiono mechanizmy, które mogą
zwiększyć skuteczność opisywanych wcześniej metod. Opisano kluczowe algorytmy
i~utworzony, proponowany klasyfikator. Opracowane zostało także podsumowanie
wyników krótkich badań przeprowadzonych nad zwiększaniem efektywności.
