\chapter*{Wstęp}
\addcontentsline{toc}{chapter}{Wstęp}
% Zadania do pracy od promotora:
% 1. Podstawy funkcjonowania internetu - założenia i ich realizacja
% 2. Problematyka prywatności i anonimowości w internecie
% 3. Przegląd metod cyfrowego odcisku palca przeglądarek internetowych i urządzeń podłączonych do internetu
% 4. Wyzwania i rozwiązania związane z metodami cyfrowego odcisku palca
% 5. Opracowanie autorskich przykładów prezentujących możliwości metod cyfrowego odcisku palca
% 6. Przeprowadzenie eksperymentów i opracowanie wyników

% Schemat wstępu:
% [x] 1. Dlaczego autor wybrał dany temat pracy?
% [x] 2. Jakie jest tło, na którym zarysowuje się dany problem?
% [x] 3. Jakie są cele badań i jak sformułowane są zadania badawcze?
% [x] 4. Dlaczego te zadania uznać można za ważne?
% [x] 5. Co jest przedmiotem badań?
% [x] 6. Jaki jest zakres analizowanych problemów, a niekiedy jaka jest cenzura czasowa?
% [x] 7. Co w pracy mogłoby się znaleźć, ale ze względów merytorycznych, logicznych czy nawet objętościowych, zostało pominięte?
% [-] 8. Jakie zastosowano metody badań i dlaczego je wybrano?
% [x] 9. Jaki jest stand wiedzy (źródeł) w danym obszarze badawczym?
% [x] 10. Z jakimi trudnymi problemami spotkał się autor przygotowując pracę?
% [x] 11. Jaka jest struktura pracy?
% [x] 12. Jak należy rozumieć i definiować niekonwencjonalne pojęcia lub skróty używane w pracy?

Temat niniejszej pracy został wybrany głównie ze względu na chęć przedstawienia
niektórych aspektów metod cyfrowego odcisku palca przeglądarek internetowych i
urządzeń podłączonych do Internetu społeczności akademickiej w Polsce:
możliwościach, zagrożeniach, wyzwaniach i rozwiązaniach z nim związanych. Autor
niniejszej pracy zauważa także powiększający się problem kryzysu prywatności w
Internecie, który często przenika także do świata fizycznego.

Identyfikacja użytkowników Internetu zaczęła być pożądana w chwili kiedy
Internet stał się masowym medium. Identyfikacja wykorzystywana jest zwykle do
śledzenia użytkowników przez platformy reklamowe w sieci, ale może być także
używana do całkowitej deanonimizacji użytkowników. Tworzenie cyfrowych odcisków
palca (\emph{fingerprinting}) to jedna z metod identyfikacji, która pozwala
serwerom WWW jednoznacznie zidentyfikować urządzenie użytkownika za pomocą
informacji wysyłanych przez to urządzenie lub przeglądarkę internetową
(działającą na tym urządzeniu) wtedy, kiedy te informacje są unikalne dla
większości z nich, tworząc ich cyfrowy odcisk palca (\emph{fingerprint}). Pisząc
bardziej obrazowo: na przykład w przypadku przeglądarek---w przeciwieństwie do
cookies lub lokalnych obiektów przechowywania cyfrowy odcisk przeglądarki
internetowej pozostaje ten sam nawet w tzw. trybie prywatnym, lub po
zresetowaniu przeglądarki do ustawień fabrycznych. Jest to zatem identyfikacja
niezależna od stanu, którą na dodatek bardzo trudno wykryć lub jest to zupełnie
niemożliwe. Biorąc pod uwagę te właściwości, jest to zupełna nowość w tej
materii. Tworzenie cyfrowych odcisków rzutuje m.in. na obszary bezpieczeństwa
komputerowego i prywatności. W historii Internetu bezpieczeństwo i prywatność
były często rozważanymi tematami; dyskusja o bezpośrednich zagrożeniach w
związku z tymi obszarami i czynne zapobieganie ich rozwojowi jest zwykle sprawą
najwyższej wagi. Celem niniejszej pracy jest zapoznanie odbiorcy z koncepcją
cyfrowych odcisków, zbadanie problematyki prywatności i anonimowości w ich
kontekście i wyróżnienie oraz szczegółowe opisanie kluczowych metod tej techniki
identyfikacji użytkowników. Rozważana jest także skuteczność tych metod i
możliwości usprawnienia ich efektywności.

W pracy skupiono się na genezie tworzenia niniejszych odcisków, zagrożeniach,
jakie ze sobą niosą, pozytywnych aspektach dostępności kolejnej metody
identyfikacji użytkowników (analizowany jest m.in. potencjał w operacjach
wojskowych w cyberprzestrzeni), podstawowych (na których bazie buduje się
bardziej skomplikowane) metodach i utworzeniu klasyfikatora, który w domyśle
powinien jeszcze bardziej zwiększyć ich efektywność. Ze względów logicznych i
objętościowych w pracy pominięto nieużywane już (zwykle dotyczące identyfikacji
urządzeń, które nie komunikują się za pomocą przeglądarki) lub bardzo
szczegółowe metody, których dokładna implementacja i działanie jest silnie
zależne od konkretnych potrzeb.

Stan wiedzy badanego obszaru jest relatywnie słaby, ale wynika to z jego
młodości. O tworzeniu cyfrowych odcisków przeglądarek internetowych i urządzeń
podłączonych do Internetu rozmawia się od około dekady. Wtedy opublikowana
zostaje pierwsza praca definiująca niektóre z podstawowych efektów, z którymi
mamy do czynienia do tej pory. Większość badań powstała ze względu na obserwację
rozwoju opisywanych metod ,,w środowisku naturalnym''; tworzenie cyfrowych
odcisków używane było przez różne firmy przed zgłębianiem tematu w środowisku
akademickim. Obszar jest zatem dosyć młody, ale liczba publikacji z nim
związanych nie przestaje rosnąć.

Problemy, które wynikły w trakcie pisania pracy to brak autorytatywnej
literatury i niewielka liczba nowych publikacji poruszających pewne szczególne
aspekty omawianego obszaru. Wynika to najpewniej ze zbyt szybko zmieniających
się środowisk, które należałoby opisywać. Bardzo wartościowe publikacje, na
które autor chciałby zwrócić uwagę ze względu na ich wysoką jakość to ostatnie
prace dotyczące \emph{fingerprintingu} autorstwa CJ Mitchella, W Li i NM
Al-Fannaha z Uniwersytetu Londyńskiego. Pewnym problemem było także testowanie
heurystycznych algorytmów klasyfikatora ze względu na potrzebę uczestnictwa w
nich sporej liczby urządzeń/użytkowników. Dzięki pomocy rodziny i znajomych
autora ten problem był mniej dotkliwy.

Praca podzielona jest na cztery rozdziały. W pierwszym rozdziale opisano jak
rozumieć i zdefiniowano niekonwencjonalne pojęcia, skróty używane w pracy.
Opisano historię, podstawy funkcjonowania Internetu (założenia i ich realizację)
i szczegółowo tło, na jakim zarysowuje się nowa, opisywana technika
identyfikacji użytkowników. W drugim rozdziale zbadano problematykę prywatności
i anonimowości w Internecie oraz wpływ na nią opisywanej techniki. Rozdział ten
zawiera także przegląd rozwiązań i wyzwań dotyczących tworzenia cyfrowych
odcisków, a także ich potencjał w innych, mniej oczywistych obszarach. Trzeci
rozdział zawiera przegląd wybranych metod i autorskie przykłady prezentujące
możliwości tych metod. Czwarty rozdział jest najbardziej eksperymentalnym, ale
także ,,produktowym'' rozdziałem. W tym rozdziale przedstawiono mechanizmy,
które mogą zwiększyć efektywność opisywanych wcześniej metod. Opisano kluczowe
algorytmy i utworzony, proponowany klasyfikator. Opracowane zostały także wyniki
przeprowadzonych badań nad zwiększaniem efektywności.
