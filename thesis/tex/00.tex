\chapter*{Wstęp}
\addcontentsline{toc}{chapter}{Wstęp}
% Zadania do pracy od promotora:
% 1. Podstawy funkcjonowania internetu - założenia i ich realizacja
% 2. Problematyka prywatności i anonimowości w internecie
% 3. Przegląd metod cyfrowego odcisku palca przeglądarek internetowych i urządzeń podłączonych do internetu
% 4. Wyzwania i rozwiązania związane z metodami cyfrowego odcisku palca
% 5. Opracowanie autorskich przykładów prezentujących możliwości metod cyfrowego odcisku palca
% 6. Przeprowadzenie eksperymentów i opracowanie wyników

% Schemat wstępu:
% 1. Dlaczego autor wybrał dany temat pracy?
% 2. Jakie jest tło, na którym zarysowuje się dany problem?
% 3. Jakie są cele badań i jak sformułowane są zadania badawcze?
% 4. Dlaczego te zadania uznać można za ważne?
% 5. Co jest przedmiotem badań?
% 6. Jaki jest zakres analizowanych problemów, a niekiedy jaka jest cenzura czasowa?
% 7. Co w pracy mogłoby się znaleźć, ale ze względów merytorycznych, logicznych czy nawet objętościowych, zostało pominięte?
% 8. Jakie zastosowano metody badań i dlaczego je wybrano?
% 9. Jaki jest stand wiedzy (źródeł) w danym obszarze badawczym?
% 10. Z jakimi trudnymi problemami spotkał się autor przygotowując pracę?
% 11. Jaka jest struktura pracy?
% 12. Jak należy rozumieć i definiować niekonwencjonalne pojęcia lub skróty używane w pracy?

Temat niniejszej pracy został wybrany głównie ze względu na chęć przedstawienia
niektórych aspektów metod cyfrowego odcisku palca przeglądarek internetowych i
urządzeń podłączonych do Internetu społeczności akademickiej w Polsce:
możliwościach, zagrożeniach, wyzwaniach i rozwiązaniach z nim związanych.
